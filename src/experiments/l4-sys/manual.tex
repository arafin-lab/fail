\documentclass[a4paper,10pt]{article}
\usepackage[utf8]{inputenc}
\usepackage{hyphenat}
\tolerance 2414
\hbadness 2414
\emergencystretch 1.5em
\hfuzz 0.3pt
\widowpenalty=10000     % Hurenkinder
\clubpenalty=10000      % Schusterjungen
\vfuzz \hfuzz
\raggedbottom
\usepackage{hyperref}

%opening
\title{L4Sys User Manual}
\author{Martin Unzner (\href{mailto:munzner@os.inf.tu-dresden.de}{munzner@os.inf.tu-dresden.de}), \\
        Björn Döbel (\href{mailto:doebel@os.inf.tu-dresden.de}{doebel@os.inf.tu-dresden.de})}    

\begin{document}

\maketitle

\begin{abstract}
This document describes how to use the L4Sys experiment suite.
However, this is not a complete documentation. When in doubt,
please read the source code or contact me. Still, I would like
you to read this whole document before investigating further.
\end{abstract}

This is the user manual on the L4Sys generic system test framework.
It provides four experiment types: GPRFlip to simulate a bit flip
in a general purpose register, RATFlip to simulate a wrong association
in the register allocation table, IDCFLip to corrupt a specific instruction
and ALUInstrFlip to modify the behaviour of the arithmetic logic unit,
so that it performs a different calculation using the same parameters.

\section{Framework Setup}

Configure Fail* as described in \texttt{doc/how-to-build.txt}. In addition,
the following CMake flags need to be set:

\begin{itemize}
\item \verb+BUILD_BOCHS = ON+
\item \verb+BUILD_X86 = ON+
\item \verb+CONFIG_BOCHS_NO_ABORT = ON+
\item \verb+CONFIG_EVENT_BREAKPOINTS = ON+
\item \verb+CONFIG_EVENT_IOPORT = ON+
\item \verb+CONFIG_SR_RESTORE = ON+
\item \verb+CONFIG_SR_SAVE = ON+
\item \verb+EXPERIMENTS_ACTIVATED = l4-sys+
\end{itemize}

Enabling \verb+CONFIG_FAST_BREAKPOINTS+ may speed up the experiment clients
significantly. Enabling \verb+CONFIG_BOCHS_NO_ABORT+ is necessary to detect
whether Bochs stopped because of a bad instruction induced by IDCFlip.
Keep in mind that this implies the risk of a deadlock
in the campaign system, because packets (i.e. experiment descriptions)
are resent if the client does not answer and finally, all clients
might fail because they tried to execute the same faulty instruction.

\section{Emulator Setup}

L4Sys fault injection experiments work with Bochs only. This is partly due to
some issues with timing --- as soon as a valid model of time in the target
emulator as well as an assembler/disassembler functionality in the Fail*
framework are established, I would recommend a backend change, as Bochs'
reliability is very limited.

To setup your system, first, you need a dedicated \texttt{bochsrc} file.  It
has proven useful to have a Bochs resource file or an independent Bochs
instance with GUI enabled for the initial testing, however the experiments are
intended to be conducted without graphical output.

Bochs should be booted using a CD image containing your setup. To obtain this
setup, first build Fiasco.OC and L4Re separately as described in their
respective \href{https://os.inf.tu-dresden.de/L4Re/build.html}{build
instructions}. Make sure your setup is running, e.g., in QEMU. Once this
works, create an ISO image using the L4 build system's 
\verb+make grub2iso E=<entry>+ command. Validate that this ISO boots and runs
in Bochs.

\section{Client Setup}

All parameters of the L4Sys experiment client can be found in the file
\texttt{experimentInfo.hpp}. Normally, it should not be necessary to change
the program flow directly. However, if something bothers you, you are always
free to take a look at \texttt{experiment.cc}, too.

\subsection{Constants}

Some values are constant throughout all steps of the preparation and also when
the workload program is run.  The most important constant is
\verb+L4SYS_BOCHS_IPS+, which has to be consistent with your \texttt{bochsrc}
setting and is used for several timely calculations in the client.

\subsection{Step 0: Determine the address space}

First, we need to find the start and end instruction addresses for our
workload program and our given experiment. For this purpose, use a
disassembler, such as \texttt{objdump} or \emph{IDA Pro}. Determine the first
and last instruction for your campaign and set  \verb+L4SYS_FUNC_ENTRY+ and
\verb+L4SYS_FUNC_EXIT+ in the header file accordingly. \verb+L4SYS_NUMINSTR+
is determined automatically in a later preparation step and can be ignored for
now.

If you want your campaign only to affect a specific address space (e.g.,
because you are only interested in faults at the application level), L4Sys
leverages Fail*'s address space filtering mechanism. To determine the address
space identifier, you will have to use Bochs'
\href{http://bochs.sourceforge.net/doc/docbook/user/internal-debugger.html}{internal
debugger} and perform the following actions:
\begin{enumerate}
  \item Compile Bochs with support for the internal debugger. This can either
        be done by configuring and rebuilding the fail client accordingly or
        using a separate Bochs installation - we don't need Fail*
        functionality here.
  \item Boot your system in Bochs. The debugger prompt (or window) will
        appear. Use the \verb+lbreak+ command to set an instruction breakpoint
        to an address in your application. (Hint: Remember you already figured
        out \verb+L4SYS_FUNC_ENTRY+ previously.)
  \item Run Bochs until the breakpoint is hit. Verify that you are in the
        right address space (instruction pointers may be similar in different
        applications as L4's BID links all programs to the same starting
        address by default).
  \item Use the \verb+creg+ command to look at the current control registers.
        Set \verb+L4SYS_ADDRESS_SPACE+ to the value of the CR3 (page table
        control) register.     
\end{enumerate}

If you are not interested in address space filtering, you may set
\verb+L4SYS_ADDRESS_SPACE+ to \verb+ANY_ADDR+. Note that in this case you will
probably encounter instruction pointers across various address spaces and may
not get the unique results you want.

\subsection{Step 1: Save the initial state of the machine}

Make sure \verb+PREPARATION_STEP+ is still set to \texttt{1}, and
you have set \verb+L4SYS_ADDRESS_SPACE+ accordingly.
Now recompile and execute the framework code again, this time with the graphical
user interface disabled. The experiment client runs until
\verb+L4SYS_FUNC_ENTRY+ is reached and then saves
the complete configuration.

\subsection{Step 2: Determine the instructions to execute}

For this part, it depends on how you want to conduct the injection
experiments. Setting \verb+L4SYS_FILTER_INSTRUCTIONS+
stores all instructions by default, and
enables the filter functionality to store only those
instructions that match the filter.
Each instruction in the trace requires
an address plus an unsigned breakpoint counter,
which means 8 bytes per instruction on a 32-bit system
and 12 bytes per instruction on a 64-bit system.

If \verb+L4SYS_FILTER_INSTRUCTIONS+ is not set, the instruction
to perform the fault injection at is determined by single-stepping
through the program from the beginning, which is quite slow.
I only recommend it for long programs, where a complete
instruction trace would require several hundred megabytes of data.

No matter which method you choose, the default implementation
of the campaign server reads the total instruction count
from \verb+L4SYS_NUMINSTR+. Thus, it is mandatory to set this
value to the number of instructions available.

To obtain this number and optionally the instruction trace,
set \verb+PREPARATION_STEP+ to \texttt{2} and recompile, then execute
the experiment client. You do not have to pass parameters to Bochs
any more, because the configuration is overwritten with the
state saved in step 1.

After the program has finished, you will get a summary on the
total of instructions executed.

If you have
\verb+L4SYS_FILTER_INSTRUCTIONS+ enabled, this is not the
value you look for; it merely claims how many
instructions have been processed at all.
To set \verb+L4SYS_NUMINSTR+ correctly, you need to look for the
number before the word \emph{accepted}, which points out how many
instructions have been accepted by the applied filter. Of course,
if no filtering is selected, these two figures should be equal.
Please contact me if that is not the case.

If \verb+L4SYS_FILTER_INSTRUCTIONS+ is disabled, you should
get a statistical output on how many of the instructions
were executed in userland and kernel space, respectively,
but the interesting figure in this case is of course the overall
sum of executed instructions.

\subsection{Step 3: Determine the correct output}

This is the easiest step: Set \verb+PREPARATION_STEP+ to \texttt{3},
recompile the client and execute it in the target directory.
It runs the complete program and logs the output. You can
check the resulting file (by default \texttt{golden.out}),
and if it does not comply with your expectations of a valid
run, you should correct the entry and exit point, the address space
or, in the worst case, your Bochs settings.

\section{Campaign Setup}

To setup the actual campaign, you need to edit \texttt{campaign.cc}.
The full language capabilities of \texttt{AspectC++} are at your hand to define
the course of your experiments; a sample covering all experiment
types at random is already provided. In the experiment client,
set \verb+PREPARATION_STEP+ to \texttt{0}, which means there is nothing more
to prepare.

After you have successfully compiled both programs, you need to
start both the campaign server (\texttt{l4-sys-server})
and the experiment client. By default, they should run on the
same machine, but you can adapt the \texttt{L4SysExperiment}
constructor in \texttt{experiment.cc} to connect the \texttt{JobClient}
to a remote server instead of \texttt{localhost}. Each experiment client processes
exactly one experiment and exits. To complete your campaign,
you should use the \texttt{client.sh} script in the \texttt{scripts}
subdirectory of Fail*.

\section{Format of the result file}

When the campaign is finished, the campaign server generates a report
file (by default called \texttt{l4sys.csv}) in a primitive CSV dialect.
The only syntax rules are that the columns are separated by commas,
that the respective data sets are separated by line breaks (\verb+\n+),
and that the cells do not contain line breaks or commas.

This section lists and describes the columns in the report generated by
the campaign server, from left to right.

\begin{enumerate}
 \item \verb+exp_type+\\
       Names the experiment that generated the return data.
       If it is none of the following, a writing error occurred:
       \begin{itemize}
        \item Unknown
        \item GPR Flip
        \item RAT Flip
        \item IDC Flip
        \item ALU Instr Flip
       \end{itemize}
       For \emph{Unknown}, a debug report should be provided.
       If not, something went completely wrong, and you should
       check the logs.
 \item \verb+injection_ip+\\
       The instruction pointer of the fault injection
       in lowercase hexadecimal notation.
       Note that the injection happens right \emph{before} this
       instruction.
 \item \verb+register+\\
       When the fault injection experiment affects a general purpose register,
       it is listed here. This column should have one of the following values;
       if it does not, a writing error occurred:
       \begin{enumerate}
        \item Unknown
        \item EAX
        \item ECX
        \item EDX
        \item EBX
        \item ESP
        \item EBP
        \item ESI
        \item EDI
       \end{enumerate}
 \item \verb+instr_offset+\\
       The offset of the executed instruction, relative to either all executed
       instructions or to all listed instructions in case you applied
       a filter (see above). This offset includes multiple runs of the same
       instruction.
       For example, this is useful when you have loops in you program and
       need a rough idea how many runs your loop had executed until
       the injection.
 \item \verb+injection_bit+\\
       The bit at which the injection was performed. This value is
       only used in GPRFlip and IDCFlip. GPRFlip inverts the bit at
       position \verb+injection_bit+ in the register, counted from the right.
       IDCFlip inverts the bit at position \verb+injection_bit+
       of the current instruction, counted from the left.    
 \item \verb+resulttype+\\
       The result of the fault injection.
       This column should have one of the following values;
       if it does not, a writing error occurred:
       \begin{enumerate}
        \item Unknown
        \item No effect
        \item Incomplete execution
        \item Crash
        \item Silent data corruption
        \item Error
       \end{enumerate}
 \item \verb+resultdata+\\
       The meaning of this field can vary for each experiment. At the moment,
       all of the experiments use it to store the last instruction
       pointer of the emulator (in decimal notation).
       This information can be used to determine when a fault
       turned into a failure.
 \item \verb+output+\\
       The output on the EIA-232 serial line generated by the workload
       program. Undisplayable or reserved characters are escaped in a
       C conformant octal manner (e.g. \verb+\033+ for the Escape character).
 \item \verb+details+\\
       This column provides various details on the experiment run,
       which may help to
       trace errors or to reconstruct the injected fault.
       ALUInstrFlip uses this column to
       provide the opcode of the new instruction.
\end{enumerate}

\section{Known bugs}

If you need support for more than one processor,
you will have to extend the code accordingly:
at the moment, when in doubt, it uses the first CPU.

\section{To Be Continued}

This is everything I consider important so far. If you still encounter
problems you may
contact me and I will try to set the record straight.
Happy experimenting! :)

\end{document}
